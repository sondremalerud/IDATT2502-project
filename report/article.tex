\documentclass[10pt]{article}

\usepackage[preprint]{tmlr}
\usepackage{graphicx}


\begin{document}
\title{Title goes here}


\author{\name Ola Nordmann \email ola@ntnu.no\\ \addr Norwegian University of Science and Technology,\\ Department of Computer Science,\\ NO-7491 Trondheim, Norway}


\maketitle


\begin{abstract}
  An abstract should be short but include: What is your paper about? Why is it important? How did you do it? What did you find? Why are your findings important?
  
  You can add a link to the corresponding software repository, if any, by using a footnote. For example: The results\footnote{The source code to reproduce the results can be found at https://gitlab.com/ola/project} show...
\end{abstract}

\section{Introduction}

Citation examples:

In \citet{krizhevsky2012imagenet}, a deep convolutional neural network was used to achieve state-of-the-art results in the ImageNet Large Scale Visual Recognition Challenge 2010. 

Machine learning models trained through backpropagation have become widely popular in the last decade
since AlexNet \citep{krizhevsky2012imagenet}.

\section{Preparations}

To get started with our project, some ground work was needed to be able to run and interact with Super Mario Bros. programatically.

Early scouting led us to nes-py, an OpenAI Gym compatible NES emulator, and gym-super-mario-bros \citep{gym-smb}, a wrapper for nes-py with specific methods to interact with Super Mario Bros. However, both these packages were built for the now deprecated OpenAI Gym and are not readily compatible with Gymnasium. For nes-py someone else had already done the work by forking and updating the package for Gymnasium, but we could not find an updated fork of the gym-super-mario-bros wrapper and ended up forking and updating it ourselves. The existing wrapper contained functionality for two different games and included game files that we would prefer not to publish, so we narrowed down the scope of the wrapper to only cover out desired game and require users to provide game files themselves.

Both the pre-exising fork of nes-py and our fork of gym-super-mario-bros are available as submodules in our project repository.


\section{Related Work}

Related previous work performed by other researchers should be presented here.

\section{Methods}

The reader, which has a background in machine learning, should be able to reproduce your results based on this section.

\section{Results and Discussion}

\section{Conclusion and Future Work}

\bibliography{article}
\bibliographystyle{tmlr}
\end{document}
