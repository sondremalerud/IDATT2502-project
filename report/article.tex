\documentclass[10pt]{article}

\usepackage[preprint]{tmlr}
\usepackage{graphicx}
\usepackage{caption}
\usepackage{subcaption}

\begin{document}
\title{Title goes here}


\author{\name Ola Nordmann \email ola@ntnu.no\\ \addr Norwegian University of Science and Technology,\\ Department of Computer Science,\\ NO-7491 Trondheim, Norway}


\maketitle


\begin{abstract}
  An abstract should be short but include: What is your paper about? Why is it important? How did you do it? What did you find? Why are your findings important?
  
  You can add a link to the corresponding software repository, if any, by using a footnote. For example: The results\footnote{The source code to reproduce the results can be found at https://gitlab.com/ola/project} show...
\end{abstract}

\section{Introduction}

Citation examples:

In \citet{krizhevsky2012imagenet}, a deep convolutional neural network was used to achieve state-of-the-art results in the ImageNet Large Scale Visual Recognition Challenge 2010. 

Machine learning models trained through backpropagation have become widely popular in the last decade
since AlexNet \citep{krizhevsky2012imagenet}.

\section{Preparations}

To get started with our project, some ground work was needed to be able to run and interact with Super Mario Bros. programatically.

Early scouting led us to nes-py, an OpenAI Gym compatible NES emulator, and gym-super-mario-bros \citep{gym-smb}, a wrapper for nes-py with specific methods to interact with Super Mario Bros. However, both these packages were built for the now deprecated OpenAI Gym and are not readily compatible with Gymnasium. For nes-py someone else had already done the work by forking and updating the package for Gymnasium, but we could not find an updated fork of the gym-super-mario-bros wrapper and ended up forking and updating it ourselves. The existing wrapper contained functionality for two different games and included game files that we would prefer not to publish, so we narrowed down the scope of the wrapper to only cover out desired game and require users to provide game files themselves.

Both the pre-exising fork of nes-py and our fork of gym-super-mario-bros are available as submodules in our project repository.


\section{Image processing}

Our machine learning model takes images of the game as its input, and the NES outputs colour images at a resolution of 240 pixels by 256 pixels. If we were to use this entire image as the input for our model, the state dimension of our Q-table would have a size of $240 \cdot 256 \cdot 3 = 184320$ which is rather large. We can reduce all these three numbers (horizontal resolution, vertical resolution and colours per pixel) by downsampling the images and by converting them from colour to grayscale.

\subsection{Grayscale conversion}

The number of methods to convert an image from colour to grayscale is practically infinite, and we theorized early on that our choice grayscale conversion method would impact model performance. We wanted to ensure that our conversion method resulted in an image where Mario, who is primarily red, would stand out from the primarily sky blue background. We decided to implement a small selection of conversions with different characteristics to benchmark them and see if they affect model performance.

\subsubsection*{Implemented grayscale conversion methods}

\begin{itemize}
    \item vxYCC$_{601}$ and vxYCC$_{709}$, two weighted sums of the red, green and blue channels that aim to be perceptually neutral. Weights are provided by a report from the International Telecommunications Union on Television Colorimetry (https://www.itu.int/dms\_pub/itu-r/opb/rep/R-REP-BT.2380-2-2018-PDF-E.pdf)
    \item RGB pixel average, a naive approach that simply averages the red, green and blue channels of the image
    \item RGB channel isolation, naive methods that simply return the red, green or blue colour channel of the image unaltered
\end{itemize}

TODO: BILDE SOM VISER SAMME BILDE BEHANDLET MED ALLE DE FORSKJELLIGE METODENE

\subsection{Downsampling}

As our image data is already represented by a NumPy array, we were able to use pre-existing NumPy methods to downsample the image with a configurable $K \times K$ kernel size, taking the average over a group of pixels to calculate the colour of the resulting pixel in the output image.

\subsection{Results}

By employing both downsampling and grayscale conversions, we were able to significantly reduce the size of the state dimension of the Q-table. By converting the image to grayscale and downsampling with a sample factor of 8, the state size was reduced to $30 \cdot 32 \cdot 1 = 960$, a reduction to 0.5\% of the original.

NOE OM YTELSE HER TAKK :)


\section{Related Work}

Related previous work performed by other researchers should be presented here.

\section{Methods}

The reader, which has a background in machine learning, should be able to reproduce your results based on this section.

\section{Results and Discussion}

\section{Conclusion and Future Work}

\bibliography{article}
\bibliographystyle{tmlr}
\end{document}
